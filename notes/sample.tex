%
% latex-sample.tex
%
% This LaTeX source file provides a template for a typical research paper.
%

%
% Use the standard article template.
%
\documentclass{article}

% The geometry package allows for easy page formatting.
\usepackage{geometry}
\geometry{letterpaper}

% Load up special logo commands.
\usepackage{doc}

% Package for formatting URLs.
\usepackage{url}

% Packages and definitions for graphics files.
\usepackage{graphicx}
\usepackage{epstopdf}
\DeclareGraphicsRule{.tif}{png}{.png}{`convert #1 `dirname #1`/`basename #1 .tif`.png}

%
% Set the title, author, and date.
%
\title{Aubree: A Practical Topological Quantum Computing Simulator}
\author{Victory Omole}
\date{5/29/2017}

%
% The document proper.
%
\begin{document}

% Add the title section.
\maketitle

% Add an abstract.
\abstract{
  Describe your paper in 100-200 words, give or take.  The command-line \texttt{wc} utility is really useful here!  This particular sample paper is meant to demonstrate a variety of \LaTeX\ directives for producing a well-structured, consistently-formatted scholarly document.  The actual content and outline may vary according to the needs of your specific research topic.
  
  Topological computing is the most ingored model of computation of computing because uing quantum gates is extemely popular because most Quantum Computer Scientists come from a clasical computing background, but the topological computing model is actually more elegant and if non-ablelian anyons can be found, this method is actually less prone to errors than the super conducting qubit methods of computing.

  State the problem: There is no open source satisfactory software framework in which i can play around and test how topological quantum computers work

  My Approach:Write software to get an idea of how toplogical computing because inaccesible concepts are accecible once the code is written.

  My solution: I wrote a program that could simulate a topological quantum computer and make it easy to study the mathematics and the braid theory that comes along with it.

  Background: I have built a couple of toy virtual machines and compilers in the past but I have started studying quantum computing fairly recently.

  Motivation: Toplogogical quantum computing is the most difficult and fault tolerant method of doing quantum comuting, so i feel like it is the most correct model
}

% Add various lists on new pages.
\pagebreak
\tableofcontents

%\pagebreak
%\listoffigures

%\pagebreak
%\listoftables

% Start the paper on a new page.
\pagebreak

%
% Body text.
%
\section{Introduction}
\label{introduction}

You will almost certainly start with an introductory description of the topic that you investigated in your assignment.  Discuss any goals, motivation, or examples of the subject; the key is to provide the reader with any information that is necessary to understand why your topic was worth investigating.  This descriptive section should also allow the reader to understand the subsequent detail sections on the subject.

The problem: There are barerly any implementations of topological qunatum computing in software, the only one I found was about using using surfaces codes for error correction.

Why is it interesting or important? It is interesting because if a topological computer that can simulate one qubit is built, It will be very easy to scale it to a limitless number of qubits. This is because topological quantum computers, unlike other method of quantum computing, is fault toloreant. This is important because researchers are making progress in finding materials that give topological phases in matter. The person who won last year's nobel prize basical found, So even though non-ableian anyons have not been found, if they are found the implecationns of the qunatum computing capabilities that can be build from it will be astounding.

Why is it hard? (E.g., why do naive approaches fail?) : Topological Quantum Computing is hard because the  is inaccesible from my point, a researcher needs to know about knot theory, absract algebra, lie groups, and quantum computing. It is an inter-diciplinary field that require a research to know high-elevel mathematics, physics, and computer science at once. This program is so that this field could be just a little more accesible.

Why hasn't it been solved before? (Or, what's wrong with previous proposed solutions? How does mine differ?)
There have been a couple of solutings such as the papers Introduction to topological quantum computing and John Peskill's Quantum Computing Lectures, But there are points in these guides that the subject matter can be a little too theoretical. It does not give the student a opportunity to get his/her hands dirty with how the actual system would be implemented.

What are the key components of my approach and results? Also include any specific limitations.

The limitations is that we are simulating a topological quantum computer with a classical computer and even if we use the IBM quantum computer, that is just 5 qubits. The main component is the interface where you can write a quantum program using quantum gates and it shows you how the anyons would be braided to make your program which is beautiful mathematically. There is a program called QTop that does it, but only respect to the error codes.

Summary of contributions:
1.Create a software the simulates the architecture of a topological quantum computer
2.Every simulation creates a PNG file that shows how the Anyons were braided to create a specific program.
3.The topological quantum computer can be program by using quantum gates, and these gates will then be compiled down to the braiding of the anyons.
4 The configuration of these anyons can be then simulated to IBM's quantum computer. This makes it so that it is possible for the topological quantum computer to be simulated using gates.




\section{Background, Preliminary, and Related Work}

Perhaps the most important functionality to learn for the paper is \LaTeX\ bibliography support.  Citations and references are handled automatically by \LaTeX\ through its companion program, \BibTeX.  All you have to do is provide a bibliography file that provides the reference information and internal keys (very much like variable names) that you use in your document.\footnote{And always remember to run \LaTeX\ \emph{at least twice} after running \BibTeX.}

\BibTeX\ supports virtually all kinds of references, including books \cite{dui,sgg,iokit,palmos}, parts of books \cite{userModeLinux}, articles \cite{nielsen:dui-review,heer-shneiderman,stackableThreads,xpkernel}, and conference proceedings \cite{ux-3d,iring,contextFileSearch,osHaskell,hibernator}, to name a few.  If not already included in your \LaTeX\ distribution, download and install the \texttt{url} package to support formatting of URLs; you can usually mention these in the \emph{note} or \emph{howpublished} fields of your \BibTeX\ file.

Like Section~\ref{introduction}, a background, preliminary, and related work section is also almost certainly needed for your paper.  In this section, describe any history, work, or projects that serve as direct contributors to the subject of your research paper.  Look at other papers in the literature to see how they organized, presented, and discussed prior work.

The Shneiderman/Plaisant text \cite{dui} provide some pointers to seminal or key works; because they made it into the textbook they aren't necessarily ``bleeding edge,'' but they likely provide the foundation for your chosen subject matter.
\pagebreak
\section{Main Content Sections}

The outline after the introductory and background, preliminary, and related work sections is more dependent on the specific subject of your research.  Remember to cite references where appropriate, organize the material so that it flows well and is clear to the reader.

\subsection{Multiple Outline Levels}

\LaTeX\ has support for up to three outline levels (\verb!\section!, \verb!\subsection!, and \verb!\subsubsection!).  It also recognizes \verb!\paragraph! and \verb!\subparagraph! directives, though those don't show up in the table of contents.  All of these directives expect a title.

Note also the use of the \verb!\verb! directive for inserting code-like labels or symbols.  It was particularly needed here so that we can include the backslash character in the text.

\subsection{Tables and Figures}

\LaTeX\ has full support for tables and figures.  Table~\ref{table-sample} shows a sample table and Figure~\ref{figure-sample} shows a sample figure.  Note the built-in support for captions and the automated numbering functionality.  Lists of tables and figures can also be automatically generated, as seen at the beginning of this document.

\begin{table}
\centering
\begin{tabular}{|c|c|c|}\hline
Column 1 & Column 2 & Column 3 \\\hline\hline
a & b & c \\
d & e & f \\
g & h & i \\\hline
\end{tabular}

\caption{A sample table}
\label{table-sample}
\end{table}

\begin{figure}
\centering
\includegraphics[width=2in]{space1.png} 

\caption{A sample figure}
\label{figure-sample}
\end{figure}

One very important thing to remember about how \LaTeX\ handles tables and figures by default: you don't have to worry about where they go exactly.  The general rule is that you insert them in the source after your first reference to them, and \LaTeX\ determines their final position.  It also makes decisions on how much page space to devote to them.  This all follows \LaTeX's overall theme of focusing on the content of your paper, and not its format.

Just so you can see a second table, Table~\ref{table-sample2} is provided.

\begin{table}
\centering
\begin{tabular}{|c|c|c|}\hline
Column 1 & Column 2 & Column 3 \\\hline\hline
a & b & c \\
d & e & f \\
g & h & i \\\hline
\end{tabular}

\caption{Another sample table}
\label{table-sample2}
\end{table}

\section{Another Section}

We're adding another section just so you can see how that looks.  Plus there are a few more \LaTeX\ features to illustrate.

\subsection{Bulleted and Numbered Lists}

\LaTeX\ is very good at providing clean lists.  Examples are shown below.

\begin{itemize}
\item Bulleted items come out properly indented and spaced, every time.

\begin{itemize}
\item Sub-bullets are a virtual no-brainer: just nest another \verb!itemize! block.
\item Note how the bullet character automatically changes too.
\end{itemize}

\item Just keep on adding \verb!\item!s\ldots

\item \ldots until you're done.
\end{itemize}

Numbered lists are almost identical, except that you specify \verb!enumerate! instead of \verb!itemize!.  List items are specified in exactly the same way (thus making it easy to change list types).

\begin{enumerate}
\item A list item
\item Another list item
\item A list item with multiple nested lists

\begin{itemize}
\item Nested lists can be of mixed types.
\item That's a lot of power and flexibility for the price of learning a handful of directives.

\begin{enumerate}
\item Like nested bullet lists, nested numbered lists also ``intelligently'' change their numbering schemes.
\item Meanwhile, all \emph{you} have to write is \verb!\item!.  \LaTeX\ does the rest.
\end{enumerate}
\end{itemize}

\item Back to your regularly scheduled list item

\end{enumerate}

\subsection{Subsection with Another Figure}

We may as well include a second figure also, shown in Figure~\ref{figure-sample2}.  The same image file is used, but note how it can be resized.  Again, observe how the positions of the tables and figures do not necessarily match their positions in the source file, reiterating the aforementioned \LaTeX\ functionality for deciding where these items go in the final document.  You provide an approximate location, and \LaTeX\ does the rest.

\begin{figure}
\centering
\includegraphics[width=1in]{space.png} 

\caption{Another sample figure}
\label{figure-sample2}
\end{figure}

\section{Conclusion}

Wrap up your paper with an ``executive summary'' of the paper itself, reiterating its subject and its major points.  If you want examples, just look at the conclusions from the literature.

\section{Future Work}
The biggest applications for Quantum Computing seem do be simulations for physics and chemistry. I could also have deep implications Material science. It would be greate if aubree contained a couple of the simulations algorithms. There should also be standard quantum algorithms like shors algorithm and grover's algorithm. Most researchers are not familiar with Lisp, so porting this program to a more accesible language like Python could be another opportunity.


\section{Acknowledgements}
I would like to thank Joesph Zambreno for guiding me in writing this report and proof-reading my work.

\section{Appendix}
I would like to thank Joesph Zambreno for guiding me in writing this report and proof-reading my work.




% Generate the bibliography.
\bibliography{latex-sample}
\bibliographystyle{unsrt}
https://github.com/jacobmarks/QTop

\end{document}
